\documentclass[UTF8,10pt,a4paper]{ctexart}
\usepackage[top=1in, bottom=1in, left=1in, right=1in]{geometry}


%画图包
%\usepackage{tikz}
\usepackage{tikz,tkz-tab}
\usetikzlibrary{shapes,arrows,backgrounds}
\usepackage{amsmath,bm,times}
\usepackage{xcolor} %代码高亮


%表格
\usepackage{tabularx,colortbl}
\newcommand{\gray}{\rowcolor[gray]{.90}} % Custom highlighting for the work experience and education sections

%字体
\usepackage{fontspec,xltxtra,xunicode}
%\defaultfontfeatures{Mapping=tex-text}
\setmonofont[Mapping={}]{Times New Roman} %让导入的代码引号直立
%\setmonofont{Consolas}
%\setromanfont{STZhongsong}
%\setmainfont{Times New Roman} 

%section
\ctexset{section/format=\Large\bfseries}

%图片导入
\usepackage{graphicx}
\usepackage{subfigure}
\usepackage{float}
\graphicspath{{./figs/}{./}{./figs2/}}

%logo & 页眉
\usepackage{fancyhdr}
\pagestyle{fancy}
\lhead{\includegraphics[scale=0.1]{BGIlogo.png}}

%附录                                                                                                                                                                                
\usepackage[titletoc]{appendix}

%超链接
%\usepackage[hyperfootnotes=true]{hyperref}
\usepackage[colorlinks,
            linkcolor=red,
            anchorcolor=blue,
            citecolor=green,
            urlcolor=blue]{hyperref}


%原文照列
\usepackage{verbatim}
\usepackage{clrscode}
\usepackage{listings} %插入代码
 
\lstset{numbers=left, %设置行号位置
	backgroundcolor=\color[RGB]{245,245,244},
	basicstyle=\footnotesize,
	showstringspaces=false,
        numberstyle=\tiny, %设置行号大小
        keywordstyle=\color{blue}, %设置关键字颜色
        commentstyle=\color[cmyk]{1,0,1,0}, %设置注释颜色
        frame=single, %设置边框格式
   %     escapeinside=``, %逃逸字符(1左面的键),用于显示中文
       basicstyle=\scriptsize\ttfamily,
        breaklines, %自动折行
        extendedchars=false, %解决代码跨页时,章节标题,页眉等汉字不显示的问题
        xleftmargin=2em,xrightmargin=2em, aboveskip=1em, %设置边距
        tabsize=4, %设置tab空格数
        showspaces=false, %不显示空格
        %belowcaptionskip=1\baselineskip,
        stringstyle=\color{orange}
        }

\title{天河可伸缩hadoop集群使用说明文档{\small (V1.0)}}
\author{\href{http://bigdata.genomics.cn/}{大数据计算组} \and 
\href{mailto:huangzhibo@genomics.cn}{黄志博}}
\date{\today}

\begin{document}
\maketitle
%\vspace{2em}
\tableofcontents
\newpage
\setlength{\parskip}{1ex plus 0.5ex minus 0.2ex}


\section{概述}
\label{sec:1}

为实现资源的高效利用,根据华大天津同事的要求,决定在天津天河集群实现hadoop集群的可伸缩性。

集群部署程序路径:/THL4/home/bgi\_gaea/software/GaeaPipeline/bin/hadoop\_cluster.py

\section{集群部署方法}
\subsection{在GaeaPipeline配置文件中设置}
须在配置文件中指定部署集群的目录等(见下图), hadoop集群会在生成脚本时自动部署并启动,在提交的流程任务完成时关闭。如果只启动了集群或出现其他集群没有成功关闭的情况,可使用集群部署程序直接进行关闭(见下一节)。
\lstinputlisting[language=Python]{source/hadoop.cfg}
\indent 以前的配置:
\lstinputlisting[language=Python]{source/old_hadoop.cfg}

\subsection{单独部署}
如果不需要使用GaeaPipeline,则需要手动启停集群,此时需要直接用到hadoop\_cluster.py程序。使用结束后一定要注意关闭hadoop集群,以免长时间占用资源\footnote{默认需一周后自动关闭}。\\
\indent 独立程序共有以下三个子命令:
\lstinputlisting[]{source/hadoop_cluster.help}
\subsubsection{start}
PARTITION指定启动hadoop集群的队列。SOURCE\_DIR指部署hadoop所需的软件和配置文件所在的目录,为/THL4/home/bgi\_gaea/software/hadoop\_source。
\lstinputlisting[]{source/start.help}
\subsubsection{restart}
\lstinputlisting[]{source/restart.help}
\subsubsection{stop}
同上


\section{任务提交}
hadoop的任务有两种提交任务的方式(以上两种情况通用):
\begin{itemize}
\item 提交至任何一个空闲节点。
\item 登陆至hadoop的master节点\footnote{根据程序提示或cluster\_dir下的cluster.json文件判断哪个是master节点},直接运行任务。
\end{itemize}

\section{任务监控}
访问监控网页的方法与原hadoop集群相同。监控网页的地址为master\_node\_ip:8088,如果master节点ip为121.104.22.0,则监控网页地址为121.104.22.0:8088 。

\end{document}
